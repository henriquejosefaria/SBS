\subsection{ Serviços de Músicas: Spotify}
\hfill

\par Um observador pode prever o estado emocional  de um utilizador apenas vendo a sua playlist, mas um algoritmo ser capaz de tal ato é algo completamente diferente. Ainda assim o Spotify acredita que é possível.
\par Embora o serviço de streaming de música seja muito bom, uma coisa que o Spotify não tem são algoritmos capazes de identificar o gosto dos seus ouvintes. 
\par No entanto, o analista de dados formado por Harvard Glenn McDonald, atualmente o "alquimista de dados" do Spotify, acredita que é possível fazer com que este aplicativo possa oferecer melhores músicas baseando-se não no histórico de cada subscrito, mas sim no grau de positividade de cada uma delas. Desta forma, ele e a equipa do Echo Nest, uma startup adquirida pela companhia sueca para melhorar o sistema de recomendações do serviço, desenvolveram um algoritmo de rede neural capaz de distinguir a diferença entre músicas tristes e alegres.
\par O algoritmo original do Spotify identifica volume, tempo, energia e compara-os a um grau de positividade. O software não leva em conta as letras, apenas o ritmo e dessa forma cometeu algumas gafes, como definir uma canção com uma letra triste, como uma canção animada por causa do ritmo, sendo que ela é exatamente o contrário. Pensando nisto, ele se encarregou de consertar a falha da rede neural do Spotify. 
\par De qualquer forma, o Spotify pretende oferecer mais e melhores recomendações ao identificar que tipo de músicas os subscritos estão a ouvir, mantê-los na plataforma e fazer com que se sintam minimamente compreendidos, ao invés de sugerir ritmos que o utilizador não consome,  ou recomendar uma canção feliz quando esse não é o estado de espírito do utilizador. Claro que identificar o comportamento do ouvinte é complicado e muitas vezes antiético, porém o serviço de streaming de música acredita que tais informações são preciosas e obviamente poderão ser revertidas em lucro. 
