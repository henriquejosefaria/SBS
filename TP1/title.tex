%
\title{Sistemas de Recomendação: Introdução e Aplicações Comerciais}
%
%\titlerunning{Abbreviated paper title}
% If the paper title is too long for the running head, you can set
% an abbreviated paper title here
%
\author{Adriana Lopes \and Diana Carrilho \and Henrique Faria \and Paulo Barbosa}
%
% First names are abbreviated in the running head.
% If there are more than two authors, 'et al.' is used.
%
\institute{Departamento de Informática, Universidade do Minho}
%
\maketitle              % typeset the header of the contribution
%
% Abstract
\begin{abstract}
Neste artigo cientifico procedeu-se ao estudo dos sistemas de recomendação iniciando-se com uma introdução ao tema abordado e uma contextualização histórica do mesmo. Seguidamente abordaram-se os cinco paradigmas diferentes: Sistemas baseados em filtros colaborativos (community based recommender systems), Sistemas de Recomendação Baseados em Conteudo (Content-based recommender systems), Sistemas de Recomendação Baseados em Conhecimento
(Knowledge-based recommender systems), Sistemas de Recomendação Demográficos (Demographic filtering), Sistemas de Recomendação hibridos/Agregados (Hybridand Ensemble-based recommender systems)~\cite{ref_book1}. Por fim referem-se dois exemplos práticos da utilização de sistemas de recomendação, na area da filmes (Netflix) e na de música (Spotify).\newline

\keywords{Sistemas de Recomendação  \and Netflix \and Spotify.}
\end{abstract}
%
%
% Introdução
\begin{center}
\normalsize{\bfseries Introdução}\hfill
\end{center}




