%
\title{Sistemas de Recomendação: Introdução e Aplicações Comerciais}
%
%\titlerunning{Abbreviated paper title}
% If the paper title is too long for the running head, you can set
% an abbreviated paper title here
%
\author{Adriana Lopes \and Diana Carrilho \and Henrique Faria \and Paulo Barbosa}
%
% First names are abbreviated in the running head.
% If there are more than two authors, 'et al.' is used.
%
\institute{Departamento de Informática, Universidade do Minho}
%
\maketitle              % typeset the header of the contribution
%
% Abstract
\begin{abstract}
Neste artigo científico procedeu-se ao estudo dos sistemas de recomendação iniciando-se com uma introdução ao tema abordado e uma contextualização histórica do mesmo. Seguidamente abordaram-se os cinco paradigmas diferentes: Sistemas baseados em filtros colaborativos (community based recommender systems), Sistemas de Recomendação Baseados em Conteúdo (Content-based recommender systems), Sistemas de Recomendação Baseados em Conhecimento
(Knowledge-based recommender systems), Sistemas de Recomendação Demográficos (Demographic filtering), Sistemas de Recomendação híbridos/Agregados (Hybridand Ensemble-based recommender systems)~\cite{ref_book1}. Por fim, referem-se dois exemplos práticos da utilização de sistemas de recomendação, na área da filmes (Netflix) e na de música (Spotify).\newline

\keywords{Sistemas de Recomendação  \and Netflix \and Spotify.}
\end{abstract}
%
%
% Introdução
\begin{center}
\normalsize{\bfseries Introdução}\hfill 
\end{center}
Em 2018, foi estimado que cerca de 51.2\% da população mundial, 3.9 mil milhões de pessoas, fossem utilizadores da internet~\cite{ref_url1} . O processo de pesquisa e obtenção de informação por parte de um utilizador deixou de ser algo trivial: Estes nem sempre eram apresentados o que inicialmente pretendiam. 
Vários serviços online desde motores de busca a lojas de comércio eletrónico aperceberam-se de que, para otimizar a experiência de cada cliente no seu serviço, teriam de ordenar a informação mostrada a cada um destes de acordo com a relevância que esta teria para ele. Tornou-se, então, evidente a necessidade de fornecer uma experiência personalizada da internet a cada utilizador.  
Os Sistemas de Recomendação (SR) surgiram como uma forma de resolver este problema. Estes são sistemas de filtragem de informação cujo objetivo é apresentar a um utilizador produtos que lhe sejam de interesse. 
Ainda que os SR possam ser aplicados a vários domínios (e-commerce, hotelaria e restauração, etc.…), neste trabalho debruçar-nos-emos sobre os seguintes temas: filmes e música. 





