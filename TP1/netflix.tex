\section{Exemplos caraterísticos}
\subsection{ Serviços de Filmes: NetFlix}
\hfill
\par O modelo de negócio da NetFlix é um serviço de subscrição que oferece recomendações personalizadas, para ajudar os clientes a encontrar as séries e filmes que lhes interessam.
\par O SR da Netflix consiste num conjunto de vários algoritmos, que servem para diferentes casos e se fundem para criar uma “experiência completa” na plataforma.
\par No caso da Netflix, os algoritmos para os sistemas de recomendação trabalham com o padrão standard $input> predição> resultado$. Como atributos de input temos: classificação, título do filme e número de estrelas que são atribuídas pelos utilizadores. As previsões de $ratings$ são calculadas com base nas informações que já existem no sistema, usando um sistema RMSE ($Root$ $Mean$ $Squared$ $Error$) onde é possível escolher quais os valores dos dados já existentes e dos dados que ainda não existem, criando assim uma recomendação \cite{ref_url1}.
\par O sistema de recomendação do Netflix é dividido em dois sistemas de organização e monitorização: o das $Metatags$ e o Comportamento do utilizador na plataforma. 
\par Tudo começa com a organização do catálogo do Netflix em categorias, subcategorias, géneros e tipos, todos sugeridos por um sistema de $tags$ abrangente e preciso. Para tal, a plataforma adopta as $metatags$, que são etiquetas que classificam todos os conteúdos disponíveis. Isto é, as metatags contêm informações que analisam cada característica dos títulos, tais como: o ano de produção, prémios, actores, directores, entre outras características. 
\par Como atributos para os algoritmos do SR da Netflix também podemos considerar as avaliações, comentários gerados pelos assinantes e até o comportamento do assinante na plataforma. Esta supervisão do comportamento do utilizador na interface do serviço abrange informações como: tempo que o utilizador ficou em cada sector da plataforma, o tipo de dispositivo onde está a visualizar os conteúdos. Posteriormente, todas estas informações são cruzadas, gerando as recomendações personalizadas para cada um dos utilizadores. Segundo a Netflix três a cada quatro vídeos assistidos no sistema só foram visualizados porque estavam na lista de recomendações. Actualmente a NetFlix tem investido no uso de redes neuronais.





