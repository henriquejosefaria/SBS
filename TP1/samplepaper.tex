% This is samplepaper.tex, a sample chapter demonstrating the
% LLNCS macro package for Springer Computer Science proceedings;
% Version 2.20 of 2017/10/04
%
\documentclass[runningheads]{llncs}
%
\usepackage{graphicx}
\usepackage[portuges]{babel}
\usepackage[T1]{fontenc}
\usepackage{verbatim}
\usepackage{float}
%Path relative to the .tex file containing the \includegraphics command
\graphicspath{ {./images/} }
% Used for displaying a sample figure. If possible, figure files should
% be included in EPS format.
%
% If you use the hyperref package, please uncomment the following line
% to display URLs in blue roman font according to Springer's eBook style:
% \renewcommand\UrlFont{\color{blue}\rmfamily}
\setcounter{secnumdepth}{6}
\renewcommand\theparagraph{\Alph{paragraph}}
 
\makeatletter
\renewcommand\paragraph{\@startsection{paragraph}{4}{\z@}%
                                      {-3.25ex\@plus -1ex \@minus -.2ex}%
                                      {0.0001pt \@plus .2ex}%
                                      {\normalfont\normalsize\bfseries}}
\renewcommand\subparagraph{\@startsection{subparagraph}{5}{\z@}%
                                      {-3.25ex\@plus -1ex \@minus -.2ex}%
                                      {0.0001pt \@plus .2ex}%
                                      {\normalfont\normalsize\bfseries}}
 
\counterwithin{paragraph}{subsubsection}
\counterwithin{subparagraph}{paragraph}
\makeatother

\begin{document}
%
\title{Sistemas de Recomendação: Introdução e Aplicações Comerciais}
%
%\titlerunning{Abbreviated paper title}
% If the paper title is too long for the running head, you can set
% an abbreviated paper title here
%
\author{Adriana Lopes \and Diana Carrilho \and Henrique Faria \and Paulo Barbosa}
%
% First names are abbreviated in the running head.
% If there are more than two authors, 'et al.' is used.
%
\institute{Departamento de Informática, Universidade do Minho}
%
\maketitle              % typeset the header of the contribution
%
% Abstract
\begin{abstract}
Neste artigo científico procedeu-se ao estudo dos sistemas de recomendação iniciando-se com uma introdução ao tema abordado e uma contextualização histórica do mesmo. Seguidamente abordaram-se os cinco paradigmas diferentes: Sistemas baseados em filtros colaborativos (community based recommender systems), Sistemas de Recomendação Baseados em Conteúdo (Content-based recommender systems), Sistemas de Recomendação Baseados em Conhecimento
(Knowledge-based recommender systems), Sistemas de Recomendação Demográficos (Demographic filtering), Sistemas de Recomendação híbridos/Agregados (Hybridand Ensemble-based recommender systems)~\cite{ref_book1}. Por fim, referem-se dois exemplos práticos da utilização de sistemas de recomendação, na área da filmes (Netflix) e na de música (Spotify).\newline

\keywords{Sistemas de Recomendação  \and Netflix \and Spotify.}
\end{abstract}
%
%
% Introdução
\begin{center}
\normalsize{\bfseries Introdução}\hfill 
\end{center}
Em 2018, foi estimado que cerca de 51.2\% da população mundial, 3.9 mil milhões de pessoas, fossem utilizadores da internet~\cite{ref_intro1} . O processo de pesquisa e obtenção de informação por parte de um utilizador deixou de ser algo trivial. Este nem sempre era apresentado com os requisitos pretendidos. Vários serviços online desde motores de busca a lojas de comércio eletrónico aperceberam-se de que, para otimizar a experiência de cada cliente no seu serviço, teriam de ordenar a informação mostrada a cada um destes de acordo com a relevância que esta teria para ele. Tornou-se, então, evidente a necessidade de fornecer uma experiência personalizada da internet a cada utilizador.  
Os Sistemas de Recomendação (SR) surgiram como uma forma de resolver este problema. Estes são sistemas de filtragem de informação cujo objetivo é apresentar a um utilizador produtos que lhe sejam de interesse. 
Ainda que os SR possam ser aplicados a vários domínios (e-commerce, hotelaria e restauração, etc.…), neste trabalho debruçar-nos-emos sobre os seguintes temas: filmes e música. 







\newpage
\hfill
% Aqui começam os capitulos abordados pelo trabalho

\section{Definição e História}

\par Há cerca de 25 anos a internet tornou-se extremamente popular nos países desenvolvidos. Nos Estados Unidos da América, entre 1994 e 2000, observou-se um período de crescimento exponencial no uso e adoção da internet conhecido como a bolha da internet (em inglês, dot-com bubble). Empresas de todos os domínios tiveram que se adaptar a este facto e moveram os seus negócios online. 

\par As recomendações são uma parte crucial da experiência digital personalizada ao utilizador por qualquer empresa. Um utilizador passa, em média, entre seis a oito horas por dia online ~\cite{ref_history1}, gerando dados relativos á sua atividade. 

\par Surgiu, então, a idade de utilizar estes na recomendação de produtos. 

\par Devido ao sucesso obtido na sua implementação, os sistemas de recomendação mantiveram-se populares desde então e têm vindo a ser refinados. A Amazon, por exemplo, estima que 35\% das suas vendas são provenientes do uso de sistemas de recomendação~\cite{ref_history2}. 

\par Sistemas de recomendação de alta qualidade podem transformar a experiência do utilizador agradável e estabelecer confiança e lealdade a longo termo. 

\par Os sistemas de recomendação evoluíram de gerarem simples colunas de itens ou artigos a construir páginas dinâmicas que amplificam diversos objetivos provenientes da análise do comportamento e histórico do utilizador. 

\section{Técnicas utilizadas por sistemas de recomendação}
\subsection{Modelos básicos dos sistemas de recomendação}

Os modelos basicos para sistemas de recomendação utilizam dois tipos de dados: interações utilizador-item como avaliações ou hábitos de compras e informações de atributos sobre utilizadores e items como o perfil respetivo.
Os métodos que usam o primeiro tipo de dados referido são chamados de Métodos de filtragem colaborativa, enquanto os restantes que utilizam o segundo dão pelo nome de Métodos de recomendação baseados em conteudo.
Os métodos de recomendação baseados em conteudo na maioria dos casos também fazem uso de matrizes de avaliações embora se foquem nas de um utilizador em vez de nas de um grupo de utilizadores.
Em Sistemas baseados em conhecimento as recomendações não são baseadas em historico de compras ou avaliações, em vez disso usam conhecimento sobre os requerimentos do utilizador.
 Alguns sistemas de recomendação combinam estes aspetos para criar sistemas hibridos. Estes sistemas combinam as forças dos vários tipos de sistemas para ter uma boa performance independentemente dos dados disponiveis.


\subsection{Modelos de filtros colaborativos}


Modelos de filtragem colaborativa usam avaliações de vários utilizadores para fazer recomendações.
O principal desafio para estes modelos reside nas matrizes esparsas.
Por exemplo, numa plataforma de música como o spotify,em que as musicas têm maior cotação dependendo do número de vezes que são ouvidas por um utilizador e em que os utilizadores só ouviram uma pequena porção de todas as músicas disponíveis, as matrizes Utilizador x Musica têm poucas avaliações visto que uma grande parte das musicas não foram ouvidas ainda.
Nestes modelos as avaliações não existentes podem servir como input visto que aquelas que existem são suficientes para estabelecer relações de similaridade entre utilizadores e items.
Por exemplo se dois utilizadores, Alice e Bob, em muitas músicas têm classificações parecidas o algoritmo pode estabelecer uma relação de semelhança entre eles. Desta forma o algoritmo pode prever que em músicas nas quais só um tenha dado classificação, o outro terá uma apreciação parecida. Esta forma de previsão consegue colmatar uma parte das classificações inesistentes na matriz.

ADICIONAR???
Furthermore, some models use carefully designed optimization techniques to create a training model in much the same way a classifier creates a training model from the labeled data. This model is then used to impute the missing values in the matrix, in the same way that a classifier imputes the missing test labels. There are two types of methods that are commonly used in collaborative filtering, which are referred to as memory-based methods and model-based methods:

\subsubsection{ Métodos Baseados em Memória}

 Estes métodos são também conhecidos como algoritmos de filtros colaborativos de vizinhança.
 Nestes, as avaliações que um utilizador pode dar a certos items é prevista com base na sua vizinhança.

Vantagens:

- Simples de implementar e as recomendações feitas são facilmente explicáveis.

Desvantagens:

- Não funcionam bem com matrizes de avaliações esparsas. Desta forma podem não haver classificações suficientes para garantir com rubustes que A vai gostar da recomendação feita.



A vizinhança para estes métodos pode ser definida de duas formas:

\paragraph{Filtros Colaborativos Baseados em Utilizadores} 

Neste caso as classificações fornecidas por utilizadores com os mesmos gostos de A são usadas para as recomendações para A. 
A ideia basica destes tipo de filtros passa por encontrar utilizadores com gostos semelhantes a A através das semelhanças entre classificações dadas por estes nas mesmas músicas. Desta forma se B tiver os mesmos gostos de A e se B der uma boa classificação a uma música C é provável que A também dê uma avaliaçõa positiva á mesma musica. Ao conjunto de utilizadores usados para inferir essas previsões dá-se o nome de vizinhança.
Funções de similaridade são computadas entre linhas da matriz para descobrir utilizadores similares.


\paragraph{Filtros Colaborativos Baseados em Items}

Para fazer previsões sobre se um utilizador A gostará de uma música B , o primeiro passo passa por determinar as X musicas mais semelhantes a B. As avaliações dadas poe A a essas X músicas é usada para inferir se A gostará da musica B.



\subsubsection{Métodos Baseados em Modelos}

Nestes modelos mineração de dados e aprendizagem máquina são usados no contexto de modelos de previsão.
Nos casos onde o modelo é parametrizado, os parâmetros são aprendidos dentro do contexto da otimização da "framework".
Alguns exemplos destes modelos incluem arvores de decisão, modelos baseados em regras, modelos de fator latente e métodos Bayesian.

Muitos deste métodos como modelos de fatores latentes conseguem solucionar problemas de matrizes esparsas.
 
Even though memory-based collaborative filtering algorithms are valued for their simplicity, they tend to be heuristic in nature, and they do not work well in all settings.



\subsection{Sistemas de Recomendação Baseados em Conteudo}

Nestes sistemas, os atributos descritivos dos items são usados para fazer recomendações.
Otermo conteudo refere-se as decrições dos items. Nestes sistemas as classificações do utilizador, bem como os seus hábitos de compras são combinados com os atributos dos items para fazer as recomendações. 
Por exemplo, num filme X o João deu uma classificação elevada mas não temos acesso a mais nenhuma classificação do mesmo filme feita por outros utilizadores. Nestas situações Modelos de filtros colaborativos não servem. No entanto a descrição do filme contem as mesmas palavras chave que outros filmes, logo estes podem ser sugeridos ao João.

Em métodos baseados em conteudo as descrições dos items, que estão marcadas com classificações são usadas para treino para criar sugestóes especificas para o utilizador. Para cada utilizador os dados usados para treino do sistema de recomendação passam pelo estórico de compras e classificações.


Vantagens:
- Quando não temos dados suficientes sobre um item, podemos inferir a futura experiencia do utilizador usando as descrições do produto e comparando-as com os produtos sobre os quais o utilizador já se pronunciou. Desta forma o sistema pode contornar a falta de dados sobre as escassas clasificações de um item.
- O utilizador novo pode especificar palavras chave no seu perfil para serem feitas recomendações com estes modelos, isto é útil em cenários "cold-start".

Desvantagens:

1. Em muitos casos são fornecidas recomendações obvias devido ás palavras chave ou conteudo. Caso o utilizador não tenha visto um determinado tipo de items ainda que goste dele estes não serão recomendados, reduzindo a diversidade dos items recomendados.

2. Não são indicados para recomendações a utilizadores novos visto que não há um histórico com o qual treinar o sistema, e para ter um sistema de recomendação rubusto é necessário um histórico considerável.



\subsection{Sistemas de Recomendação Baseados em Conhecimento}

Estes sistemas são particularmente úteis em casos em que não há histórico de compras ou avaliações, ou cenários de cold-start.
Para além disso a natureza das preferências de um utilizador pode evoluir com o tempo. Como estes modelos utilizam as preferências dos items conseguem acompanhar a tendência. 
Em outros casos pode ser dificil acompanhar as preferências de um utilizador se este só estiver interessado num atributo específico do item.
O processo de escolha de recomendações baseia-se na similaridade entre requerimentos do utilizador e descrição dos items, isto permite maior controlo do utilizador sobre o processo de recomendação podendo este explicitar o que pretende.

Estes sistemas podem ser classificados com base no tipo de interface:

\subsubsection{Sistemas de recomendação baseados em restrições}

 Utilizador especifica os atributos que pretende obter nos items que procura. Regras especificas do dominio em que se insere a busca são utilizadas. Estas regras representam o conhecimento do sistema. Os atributos (regras) adaptam-se á situação podendo servir como restrição como por exemplo, um carro que tenha sido feito  antes de 1970 não terá cruise control. Outras regras podem ser restrições no que diz respeito ao utilizador por exemplo investidores experientes não fazem investimentos de alto risco. 
 Posteriormente a uma pesquisa o utilizador pode modificar os requesitos originais, adicionando mais caso os resultados da pesquisa tenham sido muito diversos ou retirando caso os resultados não tenham sido suficientes, este processo é iterativo até o utilizador estar satisfeito com os resultados.  


\subsubsection{Sistemas de recomendação baseados em casos}

Neste tipo de sistemas o utilizador especifica casos  em vez de preferências como pontos de referência por exemplo em vez de especificar requisitos para uma casa refere uma casa que lhe agrade e procura por semelhantes. As métricas de semelhança são cuidadosamente definidas no contexto do dominio em que se inserem, formando assim a base de conhecimento do sistema. Os resultados obtidos são comummente usados como novas referências para novas pesquisas do utilizador.  Por exemplo, caso um resultado de pesquisa se assemelhe ao que o utilizador pretende, este pode selecionar esse resultado preferêncial juntamente com alguns requerimentos extra na próxima pesquisa. Este processo guia o utilizador até ao item pretendido.

Note-se que em ambos os casos, em bora de formas diferentes, o sistema fornece a oportunidade ao utilizador para mudar os requerimentos.
No primeiro, as regras(requerimentos) são usados para guiar a busca com base nas similaridades com outros artigos. Já nos sistemas baseados em casos o objeto de referência fornecido pelo utilizador é utilizado para calcular a similaridade com outros items.
 



ADICIONAR?????

How is the interactivity in knowledge-based recommender systems achieved? This guid- ance takes place through one or more of the following methods:

1. Conversational systems: In this case, the user preferences are determined iteratively in the context of a feedback loop. The main reason for this is that the item domain is complex and the user preferences can be determined only in the context of an iterative conversational system.
2. Search-based systems: In search-based systems, user preferences are elicited by using a preset sequence of questions such as the following: “Do you prefer a house in a suburban area or within the city?” In some cases, specific search interfaces may be set up in order to provide the ability to specify user constraints.
3. Navigation-based recommendation: In navigation-based recommendation, the user specifies a number of change requests to the item being currently recommended. Through an iterative set of change requests, it is possible to arrive at a desirable item. An example of a change request specified by the user, when a specific house is being recommended is as follows: “I would like a similar house about 5 miles west of the currently recommended house.” Such recommender systems are also referred to as critiquing recommender systems [417].


Os sistemas baseados em conhecimento partilham algumas das desvantagens dos sistemas baseados em conteudos. 
Por exemplo, este sistema falha no requisito de mostrar novos conteudos que o utilizador não requesite mas dos quais pode gostar.
Para além disso, este sistema não aprende com o comportamento passado do utilizador mas depende apenas dos requisitos que este inserir na pesquisa.


\subsection{Sistemas de recomendação baseado em utilidade}

Estes istemas fazem uso de funções de utilidade para computar a probabilidade do utilizador gostar do item em questão. O desafio ao aplicar estes métodos reside em definir uma função de utilizade apropriada para o utilizador visado. Estas funções podem ser vistas como conhecimento externo, logo estes sistemas podem ser vistos como casos especificos de sistemas de recomendação baseados em conhecimento, partilhando assim das suas vantagens e desvantagens.

 
\subsection{Sistemas de Recomendação Demográficos}

 Nestas técnicas a localização do utilizador alvo é utilizada para saber a propensão de uma pessoa a comprar determinado produto, por exemplo em zonas de portugal em que a água tenha muito calcário as pessoas são mais propensas a comprar detergentes para a máquina de lavar roupa que previnam o acúmulo de calcário nas ditas máquinas.
  Em muitos casos as informações demográficas podem ser combinadas com o contexto atual de informação do utilizador para guiar o proceso de recomendação.
 Embora estes sistemas por si só não forneçam as melhores recomendações, combinados com outros sistemas produzem resultados significativamente melhores.

\subsection{Sistemas de Recomendação hibridos/baseados em Conjuntos}

 Nos três principais sistemas de recomendação descritos anteriormente, podemos observar que estes funcionam bem em cenários diferentes. 
 Os sistemas com filtros colaborativos dependem das avaliações da comunidade, metodos baseados em conteúdos dependem das descrições dos items e avaliações do utilizador a quem a recomendação é feita e os sistemas baseados em conhecimento depende das interações sistema utilizador. Similarmente sistemas demográficos usam perfis demográficos para fazer recomendações.
 Como podemos observar cada uma das técnicas tem as suas vantagens e desvantagens, alguns funcionam melhor com cold-start outros quando já têm uma base sólida de avaliações do utilizador e isto leva-nos a concluir que conjugando os vários sistemas consigamos construir algoritmos mais rubustos e com melhor performance.
 Em muitos casos existe uma grande variedade de inputs o que nos dá a flexibilidade para empregar uma grande variedade de técnicas na mesma tarefa.
 Os sistemas hibridos estão extremamente relacionados com os campos da análise de conjuntos nos quais o poder de múltiplos tipos de algoritmo de aprendizagem máquina é combinado para criar um modelo mais rubusto.
 Sistemas de recomendação baseados em conjuntos são capazes de combinar não só o poder de multiplas fontes de dados como também ser altamente eficazes ao combinar multiplos modelos num só tipo. Este cenário não é muito diferente da análise de conjuntos no campo de classificação de dados.
\section{Vantagens dos Sistemas de Recomendação}

\subsection{Promotor da Recomendação}
\begin{itemize}
\item Proporciona uma melhor interação com o utilizaodr, respondendo de forma mais acertiva e direta às suas necessidades.
\item Antecipação das necessidades do utilizador~\cite{ref_book2}.
\item Redução de custos e maior previsibilidade de vendas devido ao conhecimento dos utilizadores e das suas necessidades.
\item Possibilidade de redução do stock aumentando assim a rentabilidade de negócios comerciais.
\item Facilidade da descoberta e aumento do consumo de itens menos populares pois estes são recomendados a pessoas com gostos semelhantes, aumentando assim a quantidade de aquisições~\cite{ref_book2}~\cite{ref_article1}.
\item Promoção de produtos vendidos em conjunto (pacotes). Se o sistema detetar o interesse num item por parte do utilizador e que esse item é muitas vezes vendido juntamente com outro pode apresentar o segundo item (cross-seling)~\cite{ref_book2}.
\end{itemize}


\subsection{Alvo da Recomendação}
\begin{itemize}
\item Redução do tempo de procura de informações, produtos ou conteúdos.
\item Maior assertividade e menor risco de insatisfação ao adquirir um serviço ou produto.
\item Possibilidade de poupança com a fidelização em lojas/websites em que o utilizador tem interações constantes
\item Receção de produtos/conteúdos que possam interessar ao utilizador (de forma espontânea).
\end{itemize}


\section{Exemplos caraterísticos}
\subsection{ Serviços de Filmes: NetFlix}
\par O modelo de negócio da NetFlix é o um serviço de subscrição que oferece recomendações personalizadas, para ajudar os clientes a encontrar as séries e filmes que lhes interessam.
\par O SR da Netflix consiste num conjunto de vários algoritmos, que servem para diferentes casos e se fundem para criar uma “experiência completa” na plataforma.
\par No caso Netflix os algoritmos para os sistemas de recomendação trabalham com o padrão standard $input> predição> resultado$. Como atributos de input temos: classificação, título do filme e número de estrelas que são atribuídas pelos utilizadores. As previsões de $ratings$ são calculadas com base nas informações que já existem no sistema, usando um sistema RMSE ($Root$ $Mean$ $Squared$ $Error$) onde é possível escolher quais os valores dos dados já existentes e dos dados que ainda não existem, criando assim uma recomendação \cite{ref_url1}.
\par O sistema de recomendação do Netflix é dividido em dois sistemas de organização e monitorização: o das $Metatags$ e o Comportamento do utilizador na plataforma. 
\par Tudo começa com a organização do catálogo do Netflix em categorias, subcategorias, géneros e tipos, todos sugeridos por um sistema de $tags$ abrangente e preciso. Para tal, a plataforma adopta as $metatags$, que são etiquetas que classificam todos os conteúdos disponíveis. \par Isto é, as metatags contêm informações que analisam cada característica dos títulos, tais como: o ano de produção, prémios, actores, directores, desenvolvimento narrativo, se é uma adaptação de HQ, entre outras características. 
\par Como atributos para os algoritmos do SR da Netflix também podemos considerar as avaliações e os comentários gerados pelos assinantes e o comportamento do assinante na plataforma. Esta supervisão do comportamento do utilizador na interface do serviço abrange informações como: tempo que o utilizador ficou em cada sector da plataforma, o tipo de dispositivo onde está a visualizar os conteúdos e até o horário de acesso. Todos estes dados revelam informações sobre a velocidade da conexão até aos detalhes sobre os hábitos de consumo do utilizador. \par Posteriormente, todas estas informações são cruzas, gerando assim as recomendações personalizadas para cada um dos utilizadores. Segundo a Netflix três a cada quatro vídeos assistidos no sistema só foram visualizados porque estavam na lista de recomendações. 
\par Actualmente a NetFlix tem
investido no uso de redes neuronais ( esta tecnologia diz respeito a
técnicas computacionais inspiradas na estrutura neuronal de organismos inteligentes e que adquirem conhecimento através da experiência) \cite{ref_url2}.






\subsection{ Serviços de Músicas: Spotify}

\par A Música é essencial, como a comida, por exemplo, no entanto o tipo da comida varia de acordo com o nosso estado mental. A maneira como respondemos a estímulos, independentemente de cada um deles, é estritamente pessoal e com os sons especificamente isso não é diferente. Tendemos a ouvir canções mais animadas quando estamos mais alegres e do mesmo modo preferimos as tristes quando estamos em dias maus.
\par Um observador pode prever o estado emocional  de um utilizador apenas vendo a sua playlist, mas um algoritmo ser capaz de tal ato é algo completamente diferente. Ainda assim o Spotify acredita que é possível.
\par Embora o serviço de streaming de música seja muito bom, uma coisa que o Spotify não tem é Simancol: algoritmos capazes de identificar o gosto dos seus ouvintes. 
\par No entanto o analista de dados formado por Harvard Glenn McDonald, atualmente o "alquimista de dados" do Spotify acredita que é possível fazer com que este aplicativo possa oferecer melhores músicas baseando-se não no histórico de cada subscrito, mas sim no grau de positividade de cada uma delas. Desta forma, ele e a equipa do Echo Nest, uma startup adquirida pela companhia sueca para melhorar o sistema de recomendações do serviço desenvolveram um algoritmo de rede neural capaz de distinguir a diferença entre músicas tristes e alegres.
\par O algoritmo original do Spotify identifica volume, tempo, energia e compara-os a um grau de positividade. O software não leva em conta as letras, apenas o ritmo e dessa forma cometeu algumas gafes, como definir uma canção, com uma letra triste, como uma canção animada por causa do ritmo, sendo que ela é exatamente o contrário. Pensando nisto, ele se encarregou de consertar a falha da rede neural do Spotify. Thompson combinou o algoritmo com a base de dados do Genius, uma comunidade que reúne letras de mais de 25 milhões de músicas e com isso, seu "Gloom Index" é capaz de identificar e mensurar as palavras utilizadas, combina-las com índice de positividade da pesquisa original e apresentar resultados mais refinados (por exemplo catalogar músicas com ritmo depressivo e letras felizes, e vice-versa). Ainda assim ainda não é um resultado perfeito: há dificuldades em identificar sarcasmo nas letras e lidar com negativas duplas (por exemplo Can't Stop the Feeling, de Justin Timberlake acabou por ser classificada como triste).
\par De qualquer forma, o Spotify pretende oferecer mais e melhores recomendações ao identificar que tipo de músicas os subscritos estão a ouvir, mantê-los na plataforma e fazer com que sintam minimamente compreendidos, ao invés de sugerir ritmos que o utilizador não consome,  ou recomendar uma canção feliz quando esse não é o estado de espírito do utilizador. Claro que identificar o comportamento do ouvinte é complicado e muitas vezes antiético, porém o serviço de streaming de música acredita que tais informações são preciosas e obviamente poderão ser revertidas em lucro. 



%
% ---- Bibliography ----
%
% BibTeX users should specify bibliography style 'splncs04'.
% References will then be sorted and formatted in the correct style.
%
% \bibliographystyle{splncs04}
% \bibliography{mybibliography}
%
\begin{thebibliography}{8}

\bibitem{ref_intro1}
https://www.itu.int/en/ITU-D/Statistics/Pages/stat/default.aspx
\bibitem{ref_history1}
https://thenextweb.com/contributors/2019/01/30/digital-trends-2019-every-single-stat-you-need-to-know-about-the-internet
\bibitem{ref_history2}
http://rejoiner.com/resources/amazon-recommendations-secret-selling-online/
\bibitem{ref_book1}
Author, Charu C. Aggarwal, T.: Recommender Systems: The Textbook. Publisher, Springer International Publishing AG Switzerland (2016)
\bibitem{ref_book2}
Author: Alexander Tuzhilin and Gediminas Adomavicius, T.: "Toward the Next Generation of Recommender Systems: A Survey of the State-of-the-Art and Possible Extensions", pp. 734-749, vol. 17, 2005.
\bibitem{ref_article1}
Author: Anil Poriya, Neev Patel, Rekha Sharma and Tanvi Bhagat , T.: "Non-Personalized Recommender Systems and Userbased Collaborative Recommender Systems",International Journal of Applied Information Systems, Volume 6 – November 9, March 2014
\bibitem{ref_url1} {https://help.netflix.com/en/node/100639}
\bibitem{ref_url2} {http://medialabufrj.net/blog/2019/05/sistema-de-recomendacao-netflix-algoritmos-valor-de-negocios-e-inovacao/}
\bibitem{ref_url3} {https://meiobit.com/363707/spotify-algoritmo-sendo-alimentado-para-identificar-e-diferenciar-musicas-tristes-e-alegres-meta-identificar-estado-emocional-do-ouvinte-e-recomendar-playlists-mais-adequadas/
}
\bibitem{ref_url4} {https://www.ibm.com/developerworks/br/local/data/sistemas\_recomendacao/index.html}
\bibitem{ref_url5} {http://igti.com.br/blog/como-funcionam-os-sistemas-de-recomendacao/}


\end{thebibliography}
\end{document}
