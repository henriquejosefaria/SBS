\section{Definição e História}

<<<<<<< HEAD

\par Há cerca de 25 anos a internet tornou-se extremamente popular nos países desenvolvidos. Nos Estados Unidos da América, entre 1994 e 2000, observou-se um período de crescimento exponencial no uso e adoção da internet conhecido como a bolha da internet (em inglês, dot-com bubble). Empresas de todos os domínios tiveram que se adaptar a este facto e moveram os seus negócios online. 

\par As recomendações são uma parte crucial da experiência digital personalizada ao utilizador por qualquer empresa. Um utilizador passa, em média, entre seis a oito horas por dia online ~\cite{ref_history1}, gerando dados relativos á sua atividade. 

\par Surgiu, então, a idade de utilizar estes na recomendação de produtos. 

\par Devido ao sucesso obtido na sua implementação, os sistemas de recomendação mantiveram-se populares desde então e têm vindo a ser refinados. A Amazon, por exemplo, estima que 35\% das suas vendas são provenientes do uso de sistemas de recomendação~\cite{ref_history2}. 

\par Sistemas de recomendação de alta qualidade podem transformar a experiência do utilizador agradável e estabelecer confiança e lealdade a longo termo. 

\par Os sistemas de recomendação evoluíram de gerarem simples colunas de itens ou artigos a construir páginas dinâmicas que amplificam diversos objetivos provenientes da análise do comportamento e histórico do utilizador. 
=======
Há cerca de 25 anos a internet tornou-se extremamente popular nos países desenvolvidos. Nos Estados Unidos da América, entre 1994 e 2000, observou-se um período de crescimento exponencial no uso e adoção da internet conhecido como a bolha da internet (em inglês, dot-com bubble). Empresas de todos os domínios tiveram que se adaptar a este facto e moveram os seus negócios online. 

As recomendações são uma parte crucial da experiência digital personalizada ao utilizador por qualquer empresa. Um utilizador passa, em média, entre seis a oito horas por dia online [2], gerando dados relativos á sua atividade. 

Surgiu, então, a idade de utilizar estes na recomendação de produtos. 

Devido ao sucesso obtido na sua implementação, os sistemas de recomendação mantiveram-se populares desde então e têm vindo a ser refinados. A Amazon, por exemplo, estima que 35\% das suas vendas são provenientes do uso de sistemas de recomendação. [3] 

Sistemas de recomendação de alta qualidade podem transformar a experiência do utilizador agradável e estabelecer confiança e lealdade a longo termo. 

Os sistemas de recomendação evoluíram de gerarem simples colunas de itens ou artigos a construir páginas dinâmicas que amplificam diversos objetivos provenientes da análise do comportamento e histórico do utilizador. 
>>>>>>> d99d21b8cc08042d143ffc7673822be7b576c7ec
